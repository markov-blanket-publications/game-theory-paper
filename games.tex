\documentclass{article}

% ready for submission
\usepackage{arxiv}

\usepackage[utf8]{inputenc} % allow utf-8 input
\usepackage[T1]{fontenc}    % use 8-bit T1 fonts
\usepackage{hyperref}       % hyperlinks
\usepackage{url}            % simple URL typesetting
\usepackage{booktabs}       % professional-quality tables
\usepackage{amsfonts}       % blackboard math symbols
\usepackage{nicefrac}       % compact symbols for 1/2, etc.
\usepackage{microtype}      % microtypography
\usepackage{amsmath}

\title{Revisiting the foundational axioms of game theory}

\date{December 14, 2019}

% The \author macro works with any number of authors. There are two commands
% used to separate the names and addresses of multiple authors: \And and \AND.
%
% Using \And between authors leaves it to LaTeX to determine where to break the
% lines. Using \AND forces a line break at that point. So, if LaTeX puts 3 of 4
% authors names on the first line, and the last on the second line, try using
% \AND instead of \And before the third author name.

\author{%
  Aidan Rocke\\
  \texttt{aidanrocke@gmail.com} \\
  % examples of more authors
  % \And
  % Coauthor \\
  % Affiliation \\
  % Address \\
  % \texttt{email} \\
  % \AND
  % Coauthor \\
  % Affiliation \\
  % Address \\
  % \texttt{email} \\
  % \And
  % Coauthor \\
  % Affiliation \\
  % Address \\
  % \texttt{email} \\
  % \And
  % Coauthor \\
  % Affiliation \\
  % Address \\
  % \texttt{email} \\
}

\begin{document}

\maketitle

\begin{abstract}
   Since game theory came into being in 1944 with Von Neumann's publication of 'The Theory of Games and Economic Behaviour', it has been applied to a large range of disciplines with varying degrees of success but it is worth noting its abysmal failure in its original domains of application: international relations and the field of economics. The reasons for this failure are two-fold. First, game-theoretic analyses of human relations rest upon flawed assumptions of a psychological nature that aren't constrained by psychological evidence. Second, any axiomatisation of human behaviour would need to summarise advanced knowledge of embodied cognition(i.e. complex quantitative relationships between psychology and physiology) that resists the axiomatisation process. For these reasons, the author recommends that scalable approaches to experimental psychology(rather than game theory) be used to inform domestic policy not least because internal threats are often under-estimated. 
\end{abstract}

\section{Introduction}

Since game theory came into being in 1944 with Von Neumann's publication of 'The Theory of Games and Economic Behaviour', it has been applied to a large range of disciplines ranging from economic theory to evolutionary biology and even cancer treatment [1]. In particular, it has strongly influenced the development of mathematical economics and countless applied mathematicians that develop theories of rational behaviour. On one level, game theory may be considered a triumph of the axiomatic approach to the development of the mathematical sciences, of which Von Neumann was a strong proponent. However, it is worth noting its abysmal failure in its original domains of application: international relations and the field of economics. Perhaps its biggest failure was Von Neumann's usage of game theory to justify a pre-emptive nuclear strike on the Soviet Union [5]. 

	The reasons for this failure are two-fold. First, game-theoretic analyses of human relations rest upon flawed assumptions of a psychological nature that aren't constrained by psychological evidence. Second, any axiomatisation of human behaviour would need to summarise advanced knowledge of embodied cognition(i.e. complex quantitative relationships between psychology and physiology) that resists the axiomatisation process. 
	
	Having exposed the cognitive blindspots of game theorists, I propose advances in experimental psychology as a way forward. 

\newpage

\section{The empirical validity of the axioms of game theory}

Game theory is a normative theory of human decision making where humans are considered rational if they satisfy its axioms. The theory describes rational beings as agents whose preferences may be modelled by utility functions that satisfy the Von Neumann axioms [3]: 

\textit{Completeness}: An individual has well-defined preferences. Either they prefer $A$ over $B$, $B$ over $A$, or the perceived values of $A$ and $B$ are equivalent. 

\textit{Transitivity}: If an individual prefers $A$ over $B$ and $B$ over $C$, then they prefer $A$ over $C$. 

and there are two more axioms of continuity and independence. But, the first two axioms are already sufficiently problematic. 

Empirically, individuals rarely satisfy the completeness axiom. On some days, we may prefer apples over bananas. On others, bananas over apples. Second, most humans have an order of preferences that contains contradictions so transitivity is not satisfied either. It is possible to construct many sophisticated theorems and design 'efficient' markets with these axioms but they assume the most significant part of what remains to be proved. 

The tacit assumption in these axioms is that humans are logically consistent information processing devices. However, this assumption is contradicted by
empirical evidence on a daily basis. It follows that humans are irrational according to the theory and therefore the theory is not applicable to real humans. 

On the other hand, if the value systems of individuals and communities may be approximated by utility functions, it is natural to ask how these functions behave and how they evolve over time. These are scientific questions, which concern the real world, are necessarily constrained by experimental and empirical evidence of human behaviour. 

\section{Experimental psychology as a way forward}

\subsection{Embodied cognition, or physiological constraints on human psychology}

Research in experimental psychology and robotics actually indicate that there are important physiological constraints on human psychology [6]. In fact, much of human intelligence is derived from human dexterity which has allowed us to build a large number of tools. 

If we had the bodies of jellyfish or dolphins our ability to interact with the physical world would be considerably diminished, and technology as we know it would not have developed. Furthermore, the central importance of motor control to the development of human psychology may be illustrated by noting that motor control in humans is almost entirely regulated by the cerebellum. This brain region which represents ~10\% of brain mass, contains ~80\% of the functional neurons.

Other important quantitative relationships have been established such as the relation between heart-rate variability and cognitive flexibility [7]. I mention cognitive flexibility because this faculty appears to be central to the human ability to simultaneously consider several contradictory ideas without the brain exploding. In a world as complex as ours, this appears to be an important human feature.  

\subsection{Reinforcement learning and sport psychology}

Ever since Pavlov performed his experiments on reinforcement learning, the field has made important advances in our understanding of human and animal psychology. One fundamental reason for this is that reinforcement learning theories, unlike game theory, are defined by feedback loops.

The reinforcement learning framework is both effective and simple. At a time $t$, an organism in an environment $E$ makes a decision $a_t$, obtains a reward $r_t$ in consequence, and adapts its behaviour accordingly in order to maximise expected future reward. Then the objective of reinforcement learning researchers(a psychologist or neuroscientist) is to study such behaviour. 

However, most branches of human psychology have so far ignored direct correspondences between physiology and psychology with the exception of sport psychology. In fact, sport psychology is particularly well-suited for reinforcement learning analyses. Not only does human physiology(ex. sensorimotor control) play a non-trivial role, but outcomes are always quantifiable(win/loss) and feedback loops are a key part of the training process. 

One particularly nice feature of sport psychology is that it is scalable in the sense that the majority of humans have been introduced to sport, and it is 
a natural part of human evolution. 

\newpage

\section{Discussion}

Game theory like other rationalist schools of thought, creates an artificial Cartesian dichotomy between the mind and the body. This facilitates the development of mathematical theories without paying serious attention to their empirical validity. In order to remedy this situation, I propose 
experimental psychology as a scalable and empirically-grounded alternative 
to rationalist and normative schools of human behaviour. Two key opportunities
for progress are the investigation of quantitative relationships between physiology and human psychology, and longitudinal investigations of human reinforcement learning. 

Besides those reasons given by Bertrand Russell in his 1950 speech [8], I am optimistic about the potential benefits of the research and applications of 
psychology for two reasons. First, it will help us develop a better understanding
of how psychologies develop and their physiological parameters. Second, empirical and scientific evidence indicates that human nature is not immutable. To the extent that courage, patience, flexibility and intelligence are faculties that can be developed, these should be developed. 

Finally, concerning game theory I'd like to close by saying that there is nothing more irrational than the stubborn application of a theory in domains where it is not applicable.  

\section*{References}

\small

[1] Mark Gluzman, Jacob G. Scott and Alexander Vladimirsky .Optimizing adaptive cancer therapy: dynamic programming and evolutionary game theory. Proceedings of the Royal Society. 2020.

[2] Lars Pålsson Syll. Why game theory never will be anything but a footnote in the history of social science. 2018.

[3] Von Neumann & Oskar Morgenstern. Theory of Games and Economic Behavior. 1944.

[4] Samuel J. Gershman and Nathaniel D. Daw. Reinforcement Learning and Episodic Memory in Humans and Animals: An Integrative Framework. Annual Review of Psychology. 2017.

[5] Alexander J. Field. Schelling, von Neumann, and the Event that Didn’t Occur. 2013.

[6] Rodney Brooks. Elephants don’t play chess. 2006.

[7] Lorenza S Colzato et al. Variable heart rate and a flexible mind: Higher resting-state heart rate variability predicts better task-switching. 2018.

[8] Suzana Herculano-Houzel. The human brain in numbers: a linearly scaled-up primate brain. 2009.

[9] Bertrand Russell. What Desires Are Politically Important? 1950. 

\end{document}